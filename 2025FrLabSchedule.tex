\documentclass{article}
\usepackage[dvipsnames]{xcolor}
\usepackage{pdfpages}
\usepackage{fancyhdr}
\usepackage[letterpaper]{geometry}
\usepackage[perpage,symbol*]{footmisc}

\usepackage[normalem]{ulem} % For better underlines
\usepackage{enumitem} % For more control over lists
	
\newcommand{\rd}{\uline{Reading}}
\newcommand{\pc}{\uline{Practicum}}

\usepackage[hidelinks]{hyperref}

\begin{document}
\thispagestyle{empty}
\pagestyle{fancy}
\fancyhead{}
%\fancyhead[L]{Freshman Laboratory schedule (2024--25)}
\fancyhead[C]{Observing Living Beings}
\noindent Freshman Laboratory schedule (2024--25)\footnote{This schedule is not divided by semester. But there are 58 entries, of which 7 are optional; and roughly 30 classes each semester. This suggests getting through about four days of the M\&E segment by the end of first semester.}
% Make counter for entries to be accurate about 58.

\smallskip

\begin{center}
\textbf{Observing Living Beings}
\end{center}
%\begin{enumerate}[leftmargin=!,labelindent=5pt,itemindent=-15pt]

\begin{enumerate}

\item \rd: Theophrastus, \emph{An Inquiry Concerning Plants}, Book I, chapters i--ii.

\pc: Observe and discuss magnolias in courtyard (shape of whole, branching patterns, leaves, buds). Take notes and make drawings. Prepare seeds for practicum on days 3 and 4.

\item \rd: Theophrastus, I, chapters iii--iv.

\pc: Examine magnolia buds (two kinds), stems, leaves, etc. Further examination using dissecting tools, magnifying glasses, dissecting scopes. Notes and drawings. \textsc{or}

\pc: Look at some of the other trees on campus or continue looking at magnolias.

\item \rd: Goethe, \emph{Metamorphosis of Plants,}\footnote{MIT edition preferred.} poem, Goethe’s introduction, sections I--VII.

\pc: Seed, seedling, and flower observation. Notes and drawings.

\item \rd: Goethe, sections VIII--XVIII.

\pc: Seed, seedling, and flower observation continued. Notes and drawings.

\item \rd: Aristotle, \emph{Parts of Animals}, Book I, chapters 1, 5.

\pc: Collect fish from College Creek to put in tanks. Make observations from shore.

\item \rd: Aristotle, selections, \emph{On the Movement of Animals}, \emph{On the Gait of Animals}.

\pc: Observe fish in tanks. Include invertebrates such as sea-urchins, if possible.

\item \rd: Aristotle, \emph{Parts of Animals}, Book II, chapters\ 1, 2.

\pc\ I: Observe external anatomy of fish. Notes and drawings.

\item \pc\ II: Observe internal anatomy of fish. Dissection. Notes and drawings. Start discussions of the practicum. [\href{https://sjca.sharepoint.com/:b:/r/sites/Departments/Faculty/Documents/Annapolis/Freshman\%20Lab/Schedule/Alternatives_dissection.pdf?csf=1&web=1&e=tpt1w6}{\textcolor{Maroon}{Alternatives to the dissection practica.}}]

\item \rd: Aristotle, \emph{Parts of Animals}, Book II, chapters 4, 8--12; Book III, chapters 4--6.

\pc: Continue discussions of the practicum.

\item \rd:\label{soul} Aristotle, \emph{On the Soul}, Book II, chapters 1--4. [\,and 12\,]

\item \rd: Galen, \emph{On the Usefulness of the Parts of the Body}, IV.1--6 and VI.2.

 [Some classes may read
both Galen readings on day 11, to reserve day 12 for Harvey. There is also \href{https://sjca.sharepoint.com/:b:/r/sites/Departments/Faculty/Documents/Annapolis/Freshman\%20Lab/Schedule\%20and\%20files/Galen_notes.pdf?csf=1&web=1&e=CPEIlP}{\textcolor{Maroon}{a set of helpful notes}} on Galen.]

\pc: Observation and dissection of sheep pluck. Notes and drawings.

\item \rd: Galen, \emph{On the Natural Faculties}, III.15.
	Harvey, \emph{On the Movement of the Heart},\footnote{Resource edition preferred.} Letter to Dr.\
Argent, chapters\ 1--7.

\pc: Observation and dissection of cow heart. Notes and drawings.

\item \rd: Harvey, chapters 8--14.

	\pc: Observation of chick embryos.
\item \rd: Harvey, chapters 15--17, Letter to King Charles I.

	Movie: ``William Harvey and the circulation of the blood'' (40 minutes---either in class or in FSK/Hodson Room). Discussion of observations and dissections.

\item \rd: Virchow, two lectures from \emph{Cellular Pathology}.

	\pc: Microscopic examination of different types of cells, prepared slides and fresh (onion, carrot, etc.). Notes and drawings.
\item \rd: Portmann, ``The Whole and Its Parts: The Problem of Cell Formation and Organization.'' \textsc{and/or}

\rd: Shapiro, ``Bacteria as Multi-Cellular Organisms.'' (\href{https://sjca.sharepoint.com/:b:/r/sites/Departments/Faculty/Documents/Annapolis/Freshman\%20Lab/Schedule\%20and\%20files/Shapiro_1988_scientificamerican.pdf?csf=1&web=1&e=6tDLGk}{\textcolor{Maroon}{not in manual}})

	\pc: Cell observation continued, unicellular organisms. \textsc{and/or} 
	 
	\pc: Observation of volvox algae.
	
\item \rd: Driesch, \emph{The Science and Philosophy of the Organism}, pages 1--33.

	\pc: Sea urchin fertilization and early development; notes and drawings. Observing fertilization and
	its immediate effect on the egg can be done at the beginning of the class. After discussing the reading, students can return to the microscopes, late in the class, to observe early cleavage.
	\item \rd: Driesch, pages 34--58. (\textsc{or} 34--47, 52--55)

	\pc: Sea urchin gastrula and pluteus; notes and drawings.
\item \rd: Driesch, pages 59--84. (Can be skipped if needed.)

Sea Urchin Videos (\href{https://sjca.sharepoint.com/sites/SJCLab/SitePages/Sea-Urchin-Videos.aspx?csf=1&web=1&e=TMKe6l&CID=2e377289-0a6e-4f86-9e8e-49db30a84b8e}{\textcolor{Maroon}{optional}}).\footnote{There is also a memorable ``\href{https://sjca.sharepoint.com/sites/SJCLab/_layouts/15/stream.aspx?id=\%2Fsites\%2FSJCLab\%2FShared\%20Documents\%2FFRESHMAN\%20LAB\%2FFR1\%2DObserving\%20Living\%20Beings\%2FPracticums\%2F23\%20\%2D\%20Frog\%2FThe\%20Frog\%2Em4v&referrer=StreamWebApp\%2EWeb&referrerScenario=AddressBarCopied\%2Eview\%2Ee3305199\%2D3d78\%2D4523\%2Db488\%2D99c2625c582c}{\textcolor{Maroon}{Frog Film}}'' (about 30 minutes long), set to music from \emph{The Rite of Spring}, that used to shown around the second day of Spemann.}   %Flag with URL
%https://sjca.sharepoint.com/sites/SJCLab/SitePages/Sea-Urchin-Videos.aspx?csf=1&web=1&e=TMKe6l&CID=2e377289-0a6e-4f86-9e8e-49db30a84b8e
 
\item \rd: Driesch, pages 85--109.

	\pc: Begin planaria
	experiment, observation and initial cutting. (Distribute ``Student Surgery Sheet''---\href{https://sjca.sharepoint.com/:b:/r/sites/Departments/Faculty/Documents/Annapolis/Freshman\%20Lab/Schedule\%20and\%20files/Student_surgery_sheet.pdf?csf=1&web=1&e=n1myod}{\textcolor{Maroon}{not in manual}}---with pictures of expected results deleted.) % Add second sheet to pdfs
	\item \rd: Spemann ``The Organizer-Effect in Embryonic Development.'' (Nobel lecture)
	
	% Frog film? 
	%\url{https://sjca.sharepoint.com/sites/SJCLab/_layouts/15/stream.aspx?id=%2Fsites%2FSJCLab%2FShared%20Documents%2FFRESHMAN%20LAB%2FFR1%2DObserving%20Living%20Beings%2FPracticums%2F23%20%2D%20Frog%2FThe%20Frog%2Em4v&referrer=StreamWebApp%2EWeb&referrerScenario=AddressBarCopied%2Eview%2Ee3305199%2D3d78%2D4523%2Db488%2D99c2625c582c} 
	
	% Sharepoint drive:
	%https://sjca.sharepoint.com/sites/SJCLab
	% Also link to freshman lab part

	\pc: Continued observation and experimentation with planaria.
\item \rd: Spemann, \emph{Embryonic Development and Induction}, chapters XVII, XVIII.

\pc: Planaria concluded.

\item \rd: Straus, ``The Upright Posture.''

\pc: Experiment with various gaits, etc.

\item \rd: Jonas, ``Biological Foundations of Individuality.'' (optional) \textsc{or} 
	 	
\rd: Aristotle, \emph{On the Soul}, II.1-4. (optional---repeats day \ref{soul}) \rule{1.2ex}{1.2ex} 
\end{enumerate}



\newpage

\thispagestyle{plain}
\pagestyle{fancy}
\fancyhead{}
%\fancyhead[L]{Freshman Laboratory schedule (2024--25)}
\fancyhead[C]{Measurement and Equilibrium}

%\noindent Freshman Laboratory schedule (2024--25)

%\smallskip

\begin{center}
\textbf{Measurement and Equilibrium}\footnote{The practica in the manual for this segment (and the next) usually appear in separate sections after the readings. The problems for all chapters appear in the appendix.}
\end{center}

\begin{enumerate}
\item \rd:  Aristotle, selections, \emph{Categories}.\label{cat}

\pc: Observing inanimate objects: collection of rocks and minerals.

\item \rd:  ``The Question of Measurement.'' ``Ordinal, Interval, and Ratio Scales.''\footnote{This reading is indebted to a \href{https://sjca.sharepoint.com/:b:/r/sites/Departments/Faculty/Documents/Annapolis/Freshman\%20Lab/Schedule\%20and\%20files/Stevens_TheoryScalesMeasurement_1946.pdf?csf=1&web=1&e=jRLsyL}{\textcolor{Maroon}{1946 article}} in \emph{Science} by S.\ S.\ Stevens: ``On the Theory of Scales of Measurement.''}  
% Link to Stevens article

	\pc: Exercise 1, ``Measurement of a length and a width.''
	
	\item \rd: Aristotle, \emph{On the Heavens}, Book IV, chapters 1, 3--5. (optional, \href{https://sjca.sharepoint.com/:b:/r/sites/Departments/Faculty/Documents/Annapolis/Freshman\%20Lab/Schedule\%20and\%20files/Arist_On_Heavens.pdf?csf=1&web=1&e=9RDPg3}{\textcolor{Maroon}{not in manual}})

\item \rd:  ``Measuring Weight.''

	\pc: Exercises 2--9 (These time-consuming exercises may be abbreviated by, for instance, limiting the subdivision of the \emph{baros}.)

	\item \rd: Archimedes, \emph{On the Equilibrium of Planes}, Postulates, Propositions 1--7.
	 
	``Center of Weight'' (paragraph at beginning of chapter II). 

	\pc: Exercises 1--3, ``Center of Weight.''

\item \rd:  Archimedes, \emph{Planes}, Props.\ 8--10 (read enunciations of the rest). 

\pc: Exercises 4--5, ``Cases of Equilibrium,'' ``The Law of the Lever.'' (Short practicum.) 

\item \rd:  ``Turning Power.'' ``Pulls.'' 

	\pc: Exercise 6 and Exhibit 7, ``Newton’s Wheel.'' Problems, Chapter II.

\item \rd:   Archimedes, \emph{On Floating Bodies}, Postulate and
	Propositions 1--4, enunciation of 5.\footnote{Proposition 5 is needed for Exercise 1.} 

	\pc: Exercise 1, ``Buoyancy of Floating Bodies.'' (Short practicum.)

\item \rd:  Archimedes, \emph{Floating Bodies}, Propositions 5--7. ``Buoyancy.'' ``Density.'' 

	\pc: Exercise 2, ``Buoyancy of Sinking Bodies.'' (Short practicum.)

	\pc: \emph{Alternative}: ``The Crown Problem'' and Exercise 3, ``The Crown Problem.''\footnote{There is disagreement about whether the Crown Problem is worth doing. Some tutors have read \href{https://sjca.sharepoint.com/:b:/r/sites/Departments/Faculty/Documents/Annapolis/Freshman\%20Lab/Schedule\%20and\%20files/The\%20Golden\%20Crown\%20as\%20described\%20in\%20The\%20Ten\%20Books\%20on\%20Architecture\%20by\%20Marcus\%20Vitruvius\%20Pollio.pdf?csf=1&web=1&e=6rXZMT}{\textcolor{Maroon}{Vitruvius}} or \href{https://sjca.sharepoint.com/:b:/r/sites/Departments/Faculty/Documents/Annapolis/Freshman\%20Lab/Schedule\%20and\%20files/08.5\%20Galileo\%20Little\%20Balance\%202025-01-11\%2016_59_26.pdf?csf=1&web=1&e=N4DMaE}{\textcolor{Maroon}{Galileo}} instead of the manual on the problem.}


\item \rd:  Pascal, \emph{A Treatise on the Equilibrium of Liquids}, Chapters I--III. 

	\pc: Exhibit 4, ``Equilibrium of a Single Fluid.'' Also machine for multiplying forces (two syringes connected by a tube of fluid). 

\item \rd:  Pascal, \emph{Liquids}, Chapters IV--VII. 

	\pc: Exhibit 5, ``Equilibrium of Two Fluids.'' Problems, Chapter III.

\begin{minipage}{0.9\textwidth}
\item \rd:  Pascal, \emph{Treatise on the Weight of Air}, Chapters I, II. 

	\pc:\footnote{Roman numerals after (\ref{syringe})--(\ref{syphon}) above refer to locations of the experiment in the \emph{Treatise} by
	chapter and section.}
	\begin{enumerate}[nosep]
		\item Weigh a balloon with and without air. (Chapter I, paragraph 1)
		\item Seal a syringe and try to open it. (II.I)\label{syringe}
		\item Try to pry apart two glass surfaces or suspend them in air. (II.II)
		\item Straws. (II.III)
		\item Rarify vessel of air with candle to make water rise up in inverted glass. (II.III)
		\item Invert a glass of water in a tub of water. (II.IV)
		\item Make a siphon. (II.V)\label{syphon}
	\end{enumerate}
	\end{minipage}
	% Mention student sheet here and append to schedule
\item \rd:\label{air}  Pascal, \emph{Air}, Chapters III--VI; Conclusion; Perier’s letter
and experiment, Pascal’s comment.
  
\pc: Exercise 1, ``Construction of a Barometer.'' In place of a mercury barometer, make a water barometer in the pendulum pit.
Exercise 2: ``The Experiment of Perier.'' Instead of constructing barometers, read barometers installed at the Boathouse and in the McDowell cupola; bring semi-inflated taped balloons.
The Cartesian diver. Exhibit 3: “The Constant Flow Reservoir.” 

\item Same assignment as day \ref{air}.

\item \rd:  Mariotte, \emph{Relations of Pressure and Volume of Air}. ``Pressure and Volume.'' 

	\pc: Exercise 4, ``Air is condensed\dots.'' Problems, Chapter IV.
\item \rd:  ``Sensible Heat.'' Fahrenheit, ``The Fahrenheit
	Scale.''
	
	Aristotle, \emph{Parts of Animals}, II.2, 648a20--649b7. (optional, from \emph{Observing Living Beings}) 

	\pc: Exercise 1, ``The Unmarked Mercury Thermometer.'' Exercise 2, ``The Celsius Scale.'' Exercise 3, ``The Temperature of a Mixture.''

\item \rd: Black, Extracts from \emph{Lectures on the Elements of Chemistry}, ``Equilibrium of Heat,'' ``Specific Heat.'' Reread ``Ordinal, Interval, and Ratio Scales'' and Exercise 1,
``Measurement of a Length and a Width,'' in Chapter 1 of manual. 

	\pc: Exercise 4, ``The Specific Heat Capacity of Aluminum.''

\item \rd: Black, Extracts, ``Latent Heat,'' ``Of Vapor and Vaporization.'' ``The Unit of Heat.'' 

``Variability of Specific Heat Capacities with
Temperature.'' ``Scales of Measurement for Warmth and Heat.'' (optional readings)

	\pc: Exhibit or Exercise,  Supercooling (not in manual). (Short practicum.) 
	
	Problems, Chapter V

\item \rd: Gay-Lussac, ``Investigations on the Expansion of Gases
and Vapors.'' ``The Absolute Scale of Temperature.'' ``Determination of Absolute Zero.'' ``Is
Heat a Euclidean Magnitude?''

\pc: Exercise 1, ``The Expansion of Gases by Heat.'' Problems, Chapter VI.

\item \rd: Aristotle, selections, \emph{Categories} (optional---repeats day \ref{cat}). \textsc{or}

\rd: \rule[0ex]{8mm}{0.5pt}, \emph{On Generation and Corruption}, II.[1--2], 3--4, 8. (\href{https://sjca.sharepoint.com/:b:/r/sites/Departments/Faculty/Documents/Annapolis/Freshman\%20Lab/Schedule\%20and\%20files/Arist_Gen_Corrup.pdf?csf=1&web=1&e=3vRrTS}{optional---\textcolor{Maroon}{not in manual}}) \rule{1.2ex}{1.2ex}
\end{enumerate}

\newpage

\thispagestyle{plain}
\pagestyle{fancy}
\fancyhead{}
%\fancyhead[L]{Freshman Laboratory schedule (2024--25)}
\fancyhead[C]{Constitution of Bodies}

%\noindent Freshman Laboratory schedule (2024--25)

%\smallskip

% Rules of tritration, safety rules, glossary?

\begin{center}
\textbf{Constitution of Bodies}
\end{center}

\begin{enumerate}
\item \rd: Lavoisier, Preface (Dover) or Preliminary Discourse (Green Lion).\footnote{Green Lion edition preferred. There is also a helpful note on the text at the beginning of the manual.} 

\pc:	Exhibit: ``butters,'' ``oils,'' ``flowers.''

\item \rd: Lavoisier, Chapters I and II  

	\pc: Boiling water at room temperature in a
vacuum. (optional, not in the manual)

\item \rd: Lavoisier, Chapters III and IV. ``Use of the Chemistry Laboratories.''  

	\pc: Exhibit: Oxidation of rusting iron. Experiments:
Acids and alkalies.

\item \rd: Lavoisier, Chapters V and VI  

	\pc: Experiments: Metals and nonmetal oxides.
	
\item \rd: Lavoisier, Chapters VII and VIII  

	\pc: Exhibit: Action of ``fixed
air'' on green leaves. Exhibit: Liberation of hydrogen from water vapor.

\item \rd: Lavoisier, Chapters XVI and XVII. ``Remarks on Muriatic Acid.''  

	\pc: Exhibits: 1--4 (in connection with Lavoisier’s Chapter XVI).
	
	\pc: Experiments: Reactions of metals with water and acids.

\item \rd: Dalton, Extracts from \emph{A New System of Chemical Philosophy}. 

\rd: Thomson, Extracts from \emph{System of Chemistry}.
 Appendix 2: Berthollet and Proust.  

\pc: \label{multiple}Experiment: Multiple combining proportions.\footnote{\label{reversed}The order of practica on days \ref{multiple} and \ref{definite} is sometimes reversed.} % Mention this might be out of order.

\item \rd: Gay-Lussac, ``Memoir on the Combination of Gaseous Substances with Each Other.'' 

\pc: \label{definite}Experiment: Definite combining proportions.\footref{reversed}
	
\item \rd: Avogadro, ``Essay on a Manner of Determining the Relative Masses\dots.''  

	\pc: Experiment : Weighing the
molecule of a gas.\footnote{In the current manual this experiment follows the Cannizzaro reading.}

\item \rd: Remarks on ``Molecule'' and ``Atom.''

\rd: Cannizzaro, Letter to Professor S.\ De Luca, Lectures 1--5. 

\pc: Experiment: Carbon Dioxide from Dry Ice.

\item \rd: Cannizzaro, Lecture 6  

\pc: Experiment: Weighing the water molecule.

\item \rd: Mendeleev, ``The Periodic Law of the Chemical Elements,'' Introduction, Section I.

\item \rd: Mendeleev: Section II--end.  

	\pc: Experiment: Ranking elements of a particular period.
	
\item Reflection on Constitution of Bodies segment (and perhaps the class generally). \rule{1.2ex}{1.2ex}

\end{enumerate}


%\includepdf[link,linkname=heavens,pages=-]{Attachments/Aristotle/Arist_On_Heavens.pdf}
%
%\includepdf[link,linkname=Shapiro,pages=-]{Attachments/Shapiro_1988_scientificamerican.pdf}
%
%\includepdf[link,linkname=surgery,pages=-]{Attachments/Student_surgery_sheet.pdf}
%
%\includepdf[link,linkname=Galen,pages={-},fitpaper,rotateoversize=true]{Attachments/Galen_notes.pdf}
%
%\includepdf[link,linkname=generation,pages=-]{Attachments/Aristotle/Arist_Gen_Corrup.pdf}
%
%\includepdf[link,linkname=Vitruvius,pages=-]{Attachments/The Golden Crown as described in The Ten Books on Architecture by Marcus Vitruvius Pollio.pdf}
%
%\includepdf[link,linkname=Galileo,pages=-]{Attachments/08.5 Galileo Little Balance 2025-01-11 16_59_26.pdf}
%
%\includepdf[link,linkname=Stevens,pages=-]{Attachments/Stevens_TheoryScalesMeasurement_1946.pdf}


\end{document}




